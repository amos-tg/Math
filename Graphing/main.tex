\documentclass{article}
\title{Algebraic Properties}
\author{Amos Thacker-Gwaltney}
\begin{document}
\maketitle

\section{Transformations}
\subsection{Transformation Formatting}
You can only apply transformations to visualize a graph if the expression is in transformation format. \\\\
Transformation Format: $a\cdot f(b(x-h))+k$\\\\
f: The Parent Function\\
a: Vertical Stretch\\ 
b: Compress Horizontally\\
h: Horizontal Shift\\
k: Vertical Shift\\\\
These elements can be missing but if it's not in this format you can't plot a graph with transformations.


\subsection{Horizontal Translation}
Horizontal Translation shifts the functions output along the x axis positively or negatively by some amount, subtracting from the x value pushes the value the amount subtracted in the positive direction, and adding to the x value pushes the function in the negative direction.  
\[f(x)=\frac{x-5}{2}\]
This may seem like the opposite of what should happen, but you can think about this as the x value being prolonged or pre-emptive to naturally derive the reasoning behind this. When subtracting five from x it moves the graph towards positivity five, this means that the graph starts later than normal. 


\subsection{Horizontal Stretch or Compression} 
Horizontal Stretching or Compression is caused by scaling the 'b' value of an expression in transformation format, the b value is the coefficient of the x value, and potential horizontal translation all within one quantity which is within the parent function.\\\\
Transformation Format: $a\cdot f(b(x-h))+k$\\\\

Compression is caused by: $|b|>1$\\
It's called compression because the graph is squeezed into the y-axis, and is made narrower.\\

Stretching is caused by: $0<|b|<1$\\
It's called stretching because the graph stretches away from the y-axis, and is made wider.


\subsection{Y-Axis Reflection (Vertical Reflection)}
A reflection over the y-axis, every (x, y) value becomes (-x, y). The Y-Axis reflection is caused by a negative 'b' value in transformation format:\\\\
Transformation Format: $a\cdot f(b(x-h))+k$\\\\
\[f(x)=2^{-(x-2)}\]
Here, the '-' in '-'(x-2) is causing the Y-Axis transformation.


\subsection{Vertical Stretch or Compression}
Vertical Stretching or Compression is caused by scaling the multiplier of a functions parent function; in transformation format, the value being scaled is the 'a' variable.\\\\
Transformation Format: $a\cdot f(b(x-h))+k$\\\\
\[f(x)=8\sqrt{x}+4\]\\
This square root function is vertically stretched by a factor of eight.\\\\
Stretching is caused by: $|a|>1$\\
Compression is caused by: $0<|a|<1$\\


\subsection{X-Axis Reflection}
A reflection over the x-axis, every (x, y) value becomes (x, -y). The x-axis reflection is caused by a negative 'a' value in transformation format, which is characterised by being a coefficient of the parent function:\\\\
Transformation Format: $a\cdot f(b(x-h))+k$\\\\
\[f(x)=-\sqrt{x}+4\]
The '-' in the above outside the parent function is what's causing the x-axis reflection.


\subsection{Vertical Translation}
Vertical translation is caused by the 'k' factor in function transformation format. The 'k' factor is outside of the parent function and unaffected by any scaling.\\\\
Transformation Format: $a\cdot f(b(x-h))+k$\\\\
\[f(x)=-\sqrt{x}+4\]
When the k factor is raised or lowered, the function is vertically translated along the y axis based on how much it is added to or subtracted from.\\\\ 
Raises the function: $k>0$\\  
Lowers the function: $k<0$\\


\subsection{Precedence of Transformation Graphing}
\begin{enumerate}
  \item Horizontal Translation
  \item Horizontal Stretch or Compression and Y-Axis Reflection
  \item Vertical Strectch or Compression and X-Axis Reflection
  \item Vertical Translation
\end{enumerate}



\end{document}
