\documentclass{article}
\usepackage{amsmath, amssymb}

\title{Logarithms and Exponentials}
\author{Amos Thacker-Gwaltney}

\begin{document}
\maketitle
\section{Definition of a Logarithm}
if x and b are positive real numbers such that $ b \not\equiv 1 $ then $ y = log_{b}x $ is called the logarithmic function base b, where:
\[ (y =log_{b}x) = (b^y = x) \]\\
* Given $ y = log_{b}x $, the value y is the exponent to which b must be raised to obtain x.\\\\
* The value of y is called the logarithm, b is called the base, and x is called the argument.\\\\
* The equations $ y = log_{b}x $ and $ b^y = x $ both define the same relationship between x and y. The expression $ y = log_{b}x $ is called the logarithmic form and $ x = b^y $ is called the exponential form.\\



\section{The Use Case}
Some exponential equations are impossible to solve by inspection.\\

in $ 5^x = 5 $ the solution is $ x = 1 $

however, consider $ 5^x = 20 $\\

We cannot find the value of x but we know it is between one and two. That said all exponential functions are one-to-one functions, so we can isolate x by performing the inverse operation of $5^x$.\\

The inverse operation of an exponential function, base b, is the logarithmic function base b, which is defined above.\\

\section{Converting Between Log. and Exp. Form}
Questions:

\begin{equation}
  log_{2}16 = 4
\end{equation}
\begin{equation}
  log_{10}(\frac{1}{100}) = -2
\end{equation}
\begin{equation}
  log_{7}1 = 0
\end{equation}\\\\
Answers:
\begin{equation}
  (2^4 = 16) = (log_{2}16 = 4)
\end{equation}
\begin{equation}
  (10^{-2} = \frac{1}{100}) = (log_{10}(\frac{1}{100}) = -2)
\end{equation}
\begin{equation}
  (7^0 = 1) = (log_{7}1 = 0)
\end{equation}\\

\section{Common Logarithms}
In many contexts the base of a logarithm will be excluded, this doesn't mean there is no base, it means that the base is ten. Base ten logarithms are called common logaritms. The notation for a log of base ten is the leftmost logarithm denoted below, the right log is for illustrating equivalency.
\[ (y = log_{}x) = (y = log_{10}x) \]

\section{Natural Logarithm (ln)}
The logarithmic function base e is called the natural logarithmic function. The natural logarithmic function is denoted below. You can think of the ln notation as log natural.
\[ (y = \ln x) = (log_{e}x = y) \] 

\section{Basic Properties of Logarithms}
1 * $ log_{b}1 = 0 $ because $ b^0 = 1 $\\\\
2 * $ log_{b}b = 1 $ because $ b^1 = b $\\\\
3 * $ log_{b}b^x = x $ because $ b^x = b^x $\\\\
4 * $ b^{log_{b}x} = x $ because $ log_{b}x = log_{b}x $\\

Properties 3 and 4 follow from the fact that a logarithmic function is the inverse of an exponential function of the same base.\\\\
Given $ f(x) = b^x $ and $ f^{-1}(x) = log_{b}x $\\\\
* $ (f \circ f^{-1})(x) = b^{log_{b}x} = x $ (Property 4)\\
* $ (f^{-1} \circ f)(x) = log_{b}(b^x) = x $ (Property 3)\\

\section{Product Propery of Logarithms}
By definition, $ y = log_{b}x $ is equivalent to $ b^y = x $. Becuase a logarithm is an exponent, the properties of exponents can be applied to logarithms. The first is called the product property of logarithms.\\\\
Let b, x, and y be positive real numbers where $ b \ne 1 $. Then
\[log_{b}(xy) = log_{b}x + log_{b}y \] 
The logarithm of a product equals the sum of the logarithms of the factors.\\\\
Proof:\\\\
* Let $ M = log_{b}x $, which implies $ b^M = x $\\\\
* Let $ N = log_{b}y $, which implies $ b^N = y $\\\\
* Then $ xy = b^Mb^N = b^{M+N} $\\\\
Writing the expression $xy = b^{M+N}$ in logarithmic form, we have,
\[log_{b}(xy) = M + N\]
\[log_{b}(xy) = log_{b}x+log_{b}y\]\\\\
Example:\\
\[log_{3}(3 \cdot 9)=log_{3}3+log_{3}9\]
\[log_{3}27 = 1 + 2 \]
\[3=3\]\\

\section{Quotient Property of Logarithms}
$\iff$ means if and only if\\\\
This property is derived from the quotient rule of exponents which tells us that
\[\frac{b^M}{b^N} = b^{M-N} \iff b\ne0\] . This property can be applied to logarithms due to the aforementioned relaltionship between a logarithm as an exponent value of its inverse exponential function.
Let b, x, and y be positive real numbers where $b \ne 1$. Then
\[log_{b}\frac{x}{y}=log_{b}x-log_{b}{y}\]
The logarithm of a quotient equals the difference of the numerator and the logarithm of the denomiator.\\\\
Proof:\\
\[log(\frac{1,000,000}{100})=log1,000,000-log100\]
\[log10,000=6-2\]
\[4=4\]\\

\section{Power Property of Logarithms}
The power property of exponents tells us that 
\[(b^M)^N = b^{MN}\]
The same principle can be applied to logarithms.\\

Let b and x be positive real numbers where $b\ne1$. Let p be any real number.
\[log_{b}x^p = plog_{b}x\]
Proof:\\
\[log_{2}4^2 = 2log_{2}4\]
\[log_{2}16 = 2\cdot2\]
\[4 = 4\]


\section{Graphing}
Since a logarithmic function $ y = log_{b}x $ is the inverse of the corresponding function $ y = b^x $, their graphs must be symmetric with respect to the line y = x.\\\\
The range of $ y = b^x $ is the set of positive real numbers. As expected, the domain of its inverse function $ y = log_{b}x $ is the set of positive real numbers. \\\\
When you have a logarithm of the form \\\\
a. $ f(x) = log_{2}(2x+4) $\\
b. $ g(x) = log_{2}(5-x) $\\
c. $ h(x) = log_(x^2-9) $\\

(a) solution: 
\[ f(x) = log_{2}(2x+4) \]
\[ 2x + 4 > 0 \]
\[ 2x > -4 \]
\[ x > -2 \]
The domain is $(-2, \infty)$\\
The vertical asymptote is $x = -2$\\\\
(b) solution:
\[ g(x) = log_{2}(5-x) \]
\[ 5-x > 0 \]
\[ -x > -5 \]
\[ x < 5 \]
The domain is $(-\infty, 5)$\\
The vertical asymptote is $x = -2$\\\\
(c) solution: 
\[ h(x) = log_(x^2-9) \]
\[ x^2 -9 > 0 \]
\[ (x-3)(x+3) = 0 \]
The domain is $(-\infty, -3)\cup(3, \infty)$\\
The vertical asymptotes are $x=-3$ and $x=3$\\\\

\section{Change of Base Formula}
Let a and b be positive real numbers such that $a\ne1$ and $b\ne1$. Then for any positive real number x,\\
\[log_{b}x=\frac{log_{a}x}{log_{a}b}\]\\
The change-of-base formula converts a logarithm of one base to a ratio of logarithms of a different base. For the purpose of using a calculator, we often apply the change-of-base formula with base 10 or base e.\\\\
To derive the change-of-base formula, assume that a and b are positive real numbers with $a\ne1$ and $b\ne1$. Begin by letting $y=log_{b}x$. If $y=log_{b}x$, then
\[b^y=x\]
\[log_{a}b^y=log_{a}x\]
\[y \cdot log_{a}b=log_{a}x\]
\[y=\frac{log_{a}x}{log_{a}b}\]
\[log_{b}x=\frac{log_{a}x}{log_{a}b}\]

\section{Examples}
\[3^{2y-2} = \frac{1}{3}^{y-7}\]
express the fractional base as a negative exponent.
\[3^{2y-2} = (3^{-1})^{y-7}\]
\[3^{2y-2} = 3^{-y+7}\]
\[2y-2 = -y+7\]
\[2y = -y + 9\]
\[3y = 9\]
\[y = 3\]\\

\[10^{4+8y} + 3,500 = 138,000\]
\[10^{4+8y} = 134,500\]
\[log_{10}10^{4+8y} = log_{10}134,500\]
to simplify the lvalue: 
\[log_{10}10^{4+8y} = (10^y = 10^{4+8y}) = 4+8y \]
back to the problem:
\[4+8y = log134,500\]
\[8y = log134,500-4\]
\[y = \frac{log134,500-4}{8}\]

\section{Equivalence Property of Exponential Expressions}
if b, x, and y are real numbers with $b>0$ and $n\ne1$
then $b^x = b^y$ implies that $x = y$.\\\\

if two exponential equations with the same base are equal, then their exponents must be equal.
\section{Solving Exponential Equations with Logarithms}
1: Isolate the exponential expression on one side of the equation.\\
2: Take a logarithm of the same base on both sides of the equation.\\
3: Use the power property of logarithms to "bring down" the exponent.\\
4: Solve the resulting equation.\\

\end{document}
