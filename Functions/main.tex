\documentclass{article}
\title{Functions}
\usepackage{graphicx}
\graphicspath{ {../images/} }

\author{Amos Thacker-Gwaltney}
\begin{document}
\maketitle


\section{Definition of a Function}
A function is a relation, a set of ordered pairs, of which the x-axis value of the ordered pair has only one corresponding y-axis output value.\\\\
Relation R is not a function if any of the (x, y) pairs is,\\
$f(x) = (a \lor z)$ : more than ONE possible y value\\\\
Relation R is a function if for every (x, y) pair,\\
$f(x) = y$ : just ONE y value 


\section{Function Composition}
The operator $\circ$, is the function composition symbol; the rightmost operand function is ran first, then it's output is plugged into the leftmost operand function. Undefined inputs are dictated by the composite function definition, and the rightmost operand functions undefined values. 
\[f(x)=2^x-1\]
\[g(x)=\frac{1}{x}\] 
\[(g \circ f)(x)=\frac{1}{(2^x-1)}\] 
Undefined f values: None\\
Undefined $(g \circ f)$ values: $0$\\
\[f(0)=\frac{1}{2^0-1}\]
\[f(0)=\frac{1}{1-1}\]
\[f(0)=\frac{1}{0}\]
undefined values solution set = {0}


\section{Difference Quotient}
The Difference Quotient of a function is used to measure the average rate of change of a function over a small interval.
\[f(x+h)=\frac{f(x+h)-f(x)}{h}\]


\section{Even and Odd Functions}
\subsection{Even}
A function is even if, for every x in the domain, plugging in the opposite sign of x gives the same output; i.e. the graph is symmetric over the y-axis.
\[f(-x)=f(x)\]
A quadratic is an even function:\\
\begin{center}
\includegraphics[scale=0.2]{even-function}
\end{center}

\subsection{Odd}
A function is odd if, for every x in the domain, plugging in the opposite of x gives the opposite (negative) output; i.e. the graph has $180^\circ$ rotational symmetry around the origin.
\[f(-x)=-f(x)\]
A cubic function is an odd function:\\
\begin{center}
\includegraphics[scale=0.2]{odd-function}
\end{center}

\subsection{Neither}
A function is neither even nor odd if it is neither even nor odd.\\
This logarithm is an odd function:\\
\begin{center}
\includegraphics[scale=0.2]{neither-even-nor-odd-function}
\end{center}


\section{One to One, or Injective Functions}
A function f, is one-to-one, or injective, if no two different inputs produce the same output.\\\\
For the range of the function:\\\\
if $f(a)=f(b)$, then $a=b$\\\\ 
or else it is not a one-to-one/injective function.

\subsection{The Horizontal Line Test}
A visual test for graphed functions is, if whether a horizontal line intersects two points at the graph over the range of the function, if it does, than the function is not a one to one function.\\\\
Example of a non-injective quadratic:
\begin{center}
\includegraphics[scale=0.2]{non-injective-quadratic-function}
\end{center}

\section{Inverse Functions}
A function can only have an inverse function if it is an injective/one-to-one function.\\\\
An inverse function produces the input value of another one to one function when you feed it the output value of said other one-to-one function.\\\\
given f(x) is a one-to-one function
\[f^{-1}(y)=x \Rightarrow f(x)=y\]\\
Ordered inverse function derivation steps for injective functions:
\begin{enumerate}
  \item replace f(x) with y. 
  \item substitute x with y, and y with x.
  \item solve for y
  \item replace y with $f^-1(x)$\\
\end{enumerate}
Example inverse function derivation:
\[f(x)=3x-1\]
\[y=3x-1\]
\[x = 3y-1\]
\[x+1 = 3y\]
\[f^{-1}(x)=\frac{x+1}{3}\]

\subsection{Graphing an Inverse Function}
You can graph an inverse function $f^{-1}$ can be done using just the graph of function f; you take the (x,y) coordinates of function f and swap them, the new (x, y) for $f^{-1}$ is equivalent to (y, x) of function f. 


\section{Parent Functions}
A parent function is the most basic form of a family of functions, all the other functions in that family are created by applying transformations to the parent function.

\begin{center}
\subsection{Constant Functions}
\[f(x)=c\]
\includegraphics[scale=0.2]{constant-parent-function}

\subsection{Linear or Identity Function}
\[f(x)=x\]
\includegraphics[scale=0.2]{linear-parent-function}

\subsection{Quadratic Function}
\[f(x)=x^2\]
\includegraphics[scale=0.2]{quadratic-parent-function}

\subsection{Cubic Function}
\[f(x)=x^3\]
\includegraphics[scale=0.2]{cubic-parent-function}

\subsection{Square Root Function}
\[f(x)=\sqrt{x}\]
\includegraphics[scale=0.2]{square-root-parent-function}

\subsection{Cube Root Function}
\[f(x)=\sqrt[3]{x}\]
\includegraphics[scale=0.2]{cube-root-parent-function}

\subsection{Absolute Value Function}
\[f(x)=|x|\]
\includegraphics[scale=0.2]{absolute-value-parent-function}

\subsection{Rational or Reciprocal Function}
\[f(x)=\frac{1}{x}\]
\includegraphics[scale=0.2]{rational-parent-function}

\subsection{Exponential Function}
\[f(x)=2^x\]
\includegraphics[scale=0.2]{exponential-parent-function}

\subsection{Logarithmic Function}
\[f(x)=logx\]
\includegraphics[scale=0.2]{logarithmic-parent-function}

\subsection{Greatest Integer}
\subsection{Sine}
\subsection{Cosine}
\end{center}


\section{Logarithms Functions}
\subsection{Finding the Vertical Asymptote}
Transformation format: $a\cdot f(b(x-h))+k$
The vertical asymptote is present at the h value with traditional horizontal translation sign rule application.\\\\
Example:
\[f(x)=\ln(x-5)\]
the vertical asymptote is present at $x=5$.

\subsection{Finding the Domain}
Logarithms are only defined for positive arguments, so the domain is decided by the y-axis reflection and the horizontal translation of the log.\\
Y-axis reflected functions are only defined for negative arguments because the double negatives cancel.\\
Horizontal translation applies normally to modify the domain.\\ 

\subsection{Finding the Range}
The range of a logarithm is all real numbers.


\section{Rational/Reciprocal Functions}
\subsection{Finding Horizontal Asymptotes}
n = degree of the numerator\\
m = degree of the denominator\\
$a_{x}$ = coefficient of a variable of degree x
\begin{enumerate}
  \item if $n > m$, then f has no horizontal asymptote.\\
  \item if $n < m$, then the line (y = 0) is the horizontal asymptote.\\
  \item if $n = m$, then $y=\frac{a_n}{a_m}$ is the horizontal asymptote.\\
\end{enumerate}

\subsection{Finding Vertical Asymptotes}
Vertical asymptotes are where the rational is undefined, to get this undefinition find the values where the denominator equals zero, because division by zero is undefined. There can be multiple vertical asymptotes if a function has mulitple zeros for the denominator.

\[f(x)=\frac{x^2}{(x-2)(10+x)}\]
zeros of f: 2, -10\\
therefore the vertical asymptotes are: ${2, -10}$\\

\subsection{Finding Slant and Nonlinear Asymptotes}
n = degree of the numerator\\
m = degree of the denominator\\\\
if $m + 1 = n$, there's a slant asymptote.\\
if $n > m + 1$, the end behavior is nonlinear and resembles a power function.\\\\
To find the equation of a slant or nonlinear asymptote, divide the numerator of f(x) by the denominator and determine the quotient q(x).\\ 
The equation $y = q(x)$, is an equation of the asymptote.\\
Use only the quotient for the asymptote equation, discard the remainder.

\subsection{Finding the Domain}
\[f(x)=\frac{p(x)}{q(x)}\]
The domain of a reciprocal function is all real numbers excluding the zeros of the denominator

\subsection{Finding the Range}
The range of a simple rational is all real numbers excluding the horizontal asymptotes


\section{Exponential Functions}
\subsection{Finding the Horizontal Asymptote}
The horizontal asymptote of an exponential function in relation to function transformation format is placed at the c value.\\\\
Transformation format: $a \cdot f(b(x-h))+k$

\subsection{Continuous Compounding Interest Function}
A = The final amount 
P = The principal / starting amount
e = mathematical constant
r = the interest rate
t = the time period over which interest is accumulated
\[A=Pe^{rt}\]

\subsection{Periodic Compounding Interest Function}
A = Reprents the total accumulated after t years, including the principal and the compound interest.\\
P = The principal or starting value\\
r = the interest rate\\
n = the number of compounding periods per year\\
t = the time in years, or some other time period\\
\[A=P\left(1+\frac{r}{n}\right)^{nt}\]

\section{Quadratic Functions}
\subsection{Standard Form}
The standard form of a quadratic is denoted by:
\[f(x) = ax^2 + bx + c\]

\subsection{Vertex Form}
The vertex form of a quadratic is denoted by:
\[f(x) = a(x-h)^2 + k\]

\subsection{Converting Between Standard and Vertex form}
To convert between standard and vertex form you need to complete the square.
\begin{enumerate}
  \item GCF Factor a out of the first two terms:
    \[2x^2-12x+5\]
    \[2(x^2-6x)+5\]
  \item Take the coefficient of x inside the parantheses and halve it, then square it. 
    \[\frac{-6}{2}=-3\]
    \[-3^2=9\]
  \item Add and subtract the square from step 2 inside the paranthese
    \[2(x^2-6x+9-9)+5\]
  \item Factor the perfect square trinomial
    \[2((x^2-6x+9)-9)+5\]
    \[2((x-3)^2-9)+5\]
  \item Simplify
    \[2(x-3)^2-18+5\]
    \[2(x-3)^2-13\]\\\\
\end{enumerate}
This gives a format which is easy to reason about from a transformations standpoint; it looks just like our transformation format template, $a\cdot f(b(x-h))+k$. For this particular examples here are the ordered transformations:
\begin{enumerate}
  \item Horizontal Translation: Right/+3 
  \item Vertical Stretch: factor of 2
  \item Vertical Translation: Down/-13
\end{enumerate}

\subsection{Quadratic Formula}
For any quadratic equation of the form:
\[ax^2+bx+c=0\]
The quadratic formula will give you the x values at which the quadratic equals 0:
\[\frac{-b\pm\sqrt{b^2-4ac}}{2a}=x\]
This can be used to find the zeros of a quadratic function, which allows you to factor it.

\subsection{Vertex Formulas}
If you have the vertex form of a quadratic you don't need to use these, but otherwise, this is probably faster than completing the square if you don't need to complete the square.
\begin{enumerate}
  \item X-Coordinate of the Vertex:
    \[x=-\frac{b}{2a}\]
  \item Y-Coordinate of the Vertex:
    \[y=c-\frac{b^2}{4a}\]
\end{enumerate}


\end{document}
