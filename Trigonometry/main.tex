\documentclass{article}
\usepackage{amsmath, amssymb}
\title{Trigonometry}
\author{Amos Thacker-Gwaltney}

\begin{document}
\maketitle

\section{Ray}
A ray is part of a line that consists of an endpoint and all points on the line
to one side of the endpoint.\\\\
A ray is denoted symbolically by
$\overrightarrow{PQ}$ where P is the endpoint and Q is another point on the ray
to the right of P. A ray always has one endpoint which is always written first
when naming the ray. 


\section{Angle}
An angle is formed by rotating a ray about it's endpoint. The starting position
is called the initial side of the angle, and the final position of the ray is
called the terminal side. The common endpoint is called the vertex of the angle,
and the vertex is often denoted by a capitlal letter such as A.\\\\
An angle is
notated by $\angle{A}$ or by $\angle{BAC}$, where B is the initial side, A being the
vertex, and C being the terminal side. 

\subsection{Standard Position}
An angle is said to be at standard position if it's vertex is at the origin in
the xy-plane, and the initial ray is directly on the x axis.

\subsection{Measurement}
The measure of an angle quantifies the direction and amount of rotation from the
initial ray to the terminal ray. The measure is positive if the rotation is
counterclockwise, and the measurement is negative if the rotation is clockwise.
Often, degrees are used to measure the amount of rotation, one full rotation of
a ray about its vertex is equivalent to $360^{\circ}$

\subsection{Angle Type Names}
\begin{enumerate}
  \item Right Angle:\\
    Right angles are angles at $90^{\circ}$ measure.

  \item Straight Angle:\\
    Straight angles are angles at $180^{\circ}$ measure. 

  \item Acute Angle:\\
    Acute angles are angles at $0^{\circ} < \theta < 90^{\circ}$

  \item Obtuse Angle:\\
    Obtuse angles are angles at $90^{\circ} < \theta < 180^{\circ}$\\

  \item Complementary Angles: \\
    A pair of angles which sum to $90^{\circ}$

  \item Supplementary Angles: \\ 
    A pair of angles which sum to $180^{\circ}$

  \item Coterminal Angles: \\
    Two angles in standard position with the same initial side and same terminal
    side are called coterminal angles. \\
\end{enumerate}
  
\subsection{Units}
\begin{enumerate}
  \item Minutes / Arcminutes (min or $'$): \\ 
    A singular degree can be divided into 60 equal parts; each 1/60th is referred to
    as one arcminute / minute. $\left(\frac{1}{60}\right)^{\circ}=1'\lor 1min$ 
  \item Seconds / Arcseconds (sec or $"$): \\
    A singular arcminute can be divided into 60 equal parts; each 1/60th of such
    is referred to as one arcsecond. $\left(\frac{1}{60}\right)^{'}=1''\lor 1sec$
\end{enumerate}


\section{Radian Measurement}
\subsection{Definition of a Radian}
A central angle that incercepts an arc on the circle with length equal to the
radius of the circle has a measure of one radian.

\subsection{Angle Measurement}
When two lines or rays cross a circle, the part of the circle between the
intersection points is called the intercepted arc and is often denoted by s. 

Any central angle can be measured in radians by dividing the length s of the
intercepted arc by the radius r. 

\subsection{Review}
\begin{enumerate}
  \item $\pi$:\\
    is defined as the ratio of the circumference of a circle to its diameter
    d. 
  \item Circumference:\\
    Therefore the circumference is given by $C=\pi d$ or $C=2\pi r$, where r
    is the radius of the circle, dividing this by the radius gives the
    number of radians in one revolution. The angular measure of one full
    rotation is $2\pi$ 
\end{enumerate}

\subsection{Definition of Radian Angle Measure}
The radian measure of a central angle $\theta$ subtended by an arc of length s
on a circle of radius r is given by $\theta = \frac{s}{r}$.
radian measure carries no units because it is measured as a ratio of two lengths
with the same units (the units associated with s/r "cancel"). Thus, $2\pi$
radians is simply written as $2\pi$. It is universally understood that the
measure is in radians. Sometimes the notation "rad" is included for emphasis,
but is not necessary. 

\begin{enumerate}
  \item 1 revolution: 
    $2\pi = 360^{\circ}$

  \item $\frac{1}{2}$ revolution:
    $\pi = 180^{\circ}$

  \item $\frac{1}{4}$ revolution:
    $\frac{\pi}{2} = 90^{\circ}$

  \item $\frac{3}{4}$ revolution:
    $\frac{3\pi}{2} = 270^{\circ}$
\end{enumerate}

\subsection{Converting Degrees to Radians}
To convert to radians from degrees multiply the degree by
$\frac{\pi}{180^{\circ}}$
\[x^{\circ} \cdot \frac{\pi}{180^{\circ}} = x_{rad}\]
To convert from radians to degrees multiply the degree by
$\frac{180^{\circ}}{\pi}$ \[x_{rad} \cdot \frac{180^{\circ}}{\pi}=x^{\circ}\]


\section{Trigonometric Functions}
\subsection{The Unit Circle}
The unit circle consists of all points (x,y) that satisfy the equation $x^2
+ y^2 = 1$

\subsection{Functions}
Let $P(x,y)$ be the point associated with a real number t measured along the
circumference of the unit circle from the point (1,0).
\begin{enumerate}
  \item sine:\\
    $\sin t = y$

  \item cosine:\\
    $\cos t = x$ 
  
  \item tangent:\\
    $\tan t = \frac{y}{x} (x\ne0)$

  \item cosecant:\\
    $\csc t = \frac{1}{y} (y\ne0)$

  \item secant:\\
    $\sec t = \frac{1}{x} (x\ne0)$

  \item cotangent:\\
    $\cot t = \frac{x}{y} (y\ne0)$
\end{enumerate}
If P is on the y-axis in a unit circle example, then x = 0 and the
tangent and secant functions are undefined.\\\\
If P is on the x-axis in a unit circle example, then y = 0 and the
cotangent and cosecant functions are undefined. 

\subsection{Functions of Real Numbers and Angles}
If $\theta=t$ rad, then:
\[\sin t = \sin \theta\]
\[\cos t = \cos \theta\]
\[\tan t = \tan \theta\]
\[\csc t = \csc \theta\]
\[\cot t = \cot \theta\]
\[\sec t = \sec \theta\]


\section{Using Trigonometric Functions to Determine Point Values on a Unit
Circle}
In a unit circle where the $radius=1$ you can denote the equation of the
circle using $x^2+y^2=r$, this means you can plug in $\sin t$'s or $\cos t$'s
$t$ value into the circle equation to produce the other value and thus
the coordinate set for point locations, you will have to have a qualifier for
the quadrant within which the real coordinate lies in order to
determine the real $P(x,y)$ value. 


\section{Fundamental Trigonometric Identities}
When computing the value of a trigonometric function the argument MUST be
included.
\begin{enumerate}
  \item $\sin t$ and $\csc t$ are reciprocals
    \[\csc t=\frac{1}{\sin t} \lor \sin t=\frac{1}{\csc t}\]
  
  \item $\cos t$ and $\sec t$ are reciprocals
    \[\sec t=\frac{1}{\cos t} \lor \cos t=\frac{1}{\sec t}\]

  \item $\tan t$ and $\cot t$ are reciprocals
    \[\cot t=\frac{1}{\tan t} \lor \tan t=\frac{1}{\cot t}\]

  \item $\tan t$ is the ratio of $\sin t$ and $\cos t$.
    \[\tan t=\frac{\sin t}{\cos t}\]

  \item $\cot t$ is the ratio of $\cos t$ and $\sin t$
    \[\cot t=\frac{\cos t}{\sin t}\]\\
\end{enumerate}

\section{Pythagorean Identities}
\begin{enumerate}
  \item \[\sin^2t+\cos^2t=1\]
  \item \[\tan^2t+1=\sec^2t\]
  \item \[1+\cot^2t=\csc^2t\]
\end{enumerate}

\section{Periodic Functions}
A function $f$ is periodic if $f(t+p)=f(t)$ for some constant $p$.\\
The smallest positive value $p$ for which $f$ is periodic is called the
period of $f$.

\subsection{Periodic Properties of Trigonometric Functions}
The values of the six trigonometric functions of t are determined by
the corresponding point $P(x,y)$ on the unit circle. The Circumference of the
unit circle is $2\pi$, adding (or subtracting) $2\pi$ to $t$ results
in the same terminal point $(x,y)$. Consequently, the values of the
trigonometric functions are the same for $t$ and $t+2n\pi$
\begin{enumerate}
  \item Sine:\\
    $period=2\pi$\\
    $property=\sin(t+2\pi)=\sin t$\\
  
  \item Cosine:\\
    $period=2\pi$\\
    $property=\cos(t+2\pi)=\cos t$\\

  \item Cosecant:\\
    $period=2\pi$\\
    $property=\csc(t+2\pi)=\csc t$\\

  \item Secant:\\
    $period=2\pi$\\
    $property=\sec(t+2\pi)=\sec t$\\

  \item Tangent:\\
    $period=\pi$\\
    $property=\tan(t+\pi)=\tan t$\\

  \item Cotangent:\\
    $period=\pi$\\
    $property=\cot(t+\pi)=\cot t$\\
\end{enumerate}
If the period is $p$, then $f(t+p)=f(t)$. It is also true that $f(t+np)=f(t)$
for any integer $n$. That is, adding any integer multiple of the period to a
domain element of a periodic function results in the same function value.


\section{Even and Odd Properties of Trigonometric Functions}
\begin{enumerate}
  \item Sine:\\
    $\sin t=y$ and $\sin(-t)=-y$\\
    Odd Function: $\sin(-t)=-\sin t$

  \item Cosine:\\
    $\cos t=x$ and $\cos(-t)=x$\\
    Even Function: $\cos(-t)=\cos t$

  \item Cosecant:\\
    $\csc t=\frac{1}{y}$ and $\csc(-t)=\frac{1}{-y}$\\
    Odd Function: $\csc(-t)=-\csc t$

  \item Secant:\\
    $\sec t=\frac{1}{x}$ and $\sec(-t)=\frac{1}{x}$\\
    Even Function: $\sec(-t)=\sec t$

  \item Tangent:\\
    $\tan t=\frac{y}{x}$ and $\tan(-t)=\frac{-y}{x}$\\
    Odd Function: $\tan(-t)=-\tan t$

  \item Cotangent:\\
    $\cot t=\frac{x}{y}$ and $\cot(-t)=\frac{x}{-y}$\\
    Odd Function: $\cot(-t)=-\cot t$\\
\end{enumerate}

\section{Trigonometric Functions of Any Angle}
Let $\theta$ be an angle in standard position with point $P(x,y)$ on the
terminal side, and let $r=\sqrt{x^2+y^2}\ne0$ represent the distance from
$P(x,y)$ to $(0,0)$. Then:
\[\sin\theta=\frac{y}{r}\]
\[\cos\theta=\frac{x}{r}\]
\[\tan\theta=\frac{y}{x}(x\ne0)\]
\[\csc\theta=\frac{r}{y}(y\ne0)\]
\[\sec\theta=\frac{r}{x}(x\ne0)\]
\[\cot\theta=\frac{x}{y}(y\ne0)\]


\section{Reference Angles}
\subsection{Definition}
Let $\theta$ be an angle in standard position. The reference angle for
$\theta$ is the acute angle $\theta'$ formed by the terminal side of $\theta$
and the horizontal axis.\\ 
The length of $\angle\theta'$s vertical leg is $|y|$.\\
The length of the
horizontal leg is $|x|$.\\
The hypotenuse's length is $r =\sqrt{x^2+y^2}$

\subsection{Evaluate Trigonometric Functions using Reference Angles}
$\angle\theta =$ standard position\\
$\angle\theta'=$ reference angle complementary to $\angle\theta$
\[\cos \theta = \frac{x}{r} \land
\cos\theta'=\frac{adj}{hyp}=\frac{|x|}{r}\]

To Find the value of a trigonometric function of a given angle $\theta$
\begin{enumerate}
  \item Determine the function value of the reference angle $\theta'$

  \item 
    Affix the appropriate sign based on the quadrant in which $\theta$ lies
\end{enumerate}

\section{Right Triangle Trigonometry 4.3}
\subsection{Anatomy of a Triangle}
Consider a right triangle, the longest side of the triangle is called the
hypotenuse, the remaining two legs are determined relative to the position of
the considered angle. There is an acute angle $\theta$, the leg opposite of this
angle is considered the opposite leg, the adjacent leg is, similarly, considered
the adjacent leg. 

The six trigonometric functions take inputs relative to the acute angle $\theta$
to determine the ratios of the lengths of the sides of the triangle.

\subsection{Definition of Trigonometric Functions of Acute Angles}
\[sine = \sin\theta=\frac{opp}{hyp}\] 
\[cosine = \cos\theta=\frac{adj}{hyp}\]
\[tangent = \tan\theta=\frac{opp}{adj}\]
\[cosecant = \csc\theta=\frac{hyp}{opp}\]
\[secant = \sec\theta=\frac{hyp}{adj}\]
\[cotangent = \cot\theta=\frac{adj}{opp}\]

The mnemonic device "SOH-CAH-TOA" may help you remember the ratios for
$\sin\theta$, $\cos\theta$, $\tan\theta$, respectively. 

\subsection{Pythagorean Theorem}
legs a, b;\\
hypotenuse c;\\
\[a^2+b^2=c^2\]

\subsection{Cofunction Identities}
Cofunctions of complementary angles are equal; the "co" prefixing any one of
these functions stands for "complementary".

\begin{enumerate}
\item Sine and Cosine are Cofunctions:
  \[\sin\theta=\cos(90^{\circ}-\theta)\]
  \[\cos\theta=\sin(90^{\circ}-\theta)\]

\item Tangent and Cotangent are Cofunctions:
  \[\tan\theta=\cot(90^{\circ}-\theta)\]
  \[\cot\theta=\tan(90^{\circ}-\theta)\]

\item Secant and Cosecant are Cofunctions:
  \[\sec\theta=\csc(90^{\circ}-\theta)\]
  \[\csc\theta=\sec(90^{\circ}-\theta)\]

\end{enumerate}

\end{document}
