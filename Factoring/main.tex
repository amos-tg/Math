\documentclass{article}
\usepackage{amsmath, amssymb, enumitem, array}
\title{Factoring}
\author{Amos Thacker-Gwaltney}

\begin{document}
\maketitle


\section{Greatest Common Factor}
Applies the reverse of the distributive property to pull out factors from an expression. An example with variables:
\[x(x+1)=x^2+x\]
another with just numbers:
\[10(10+1)=100+10\]
this method has the highest precedence of all the factoring methods.


\section{Special Methods}
These methods have the second highest precedence, neither takes precendence over the other so inner categorical precedence is based on left to right occurrence.

\subsection{Difference of Squares}
A method used to factor a binomial group that is the difference between two perfect squares; this only holds true for subtractive square pairs because the conjugate has to cancel out the first degree values of the variable.
The Formula:
\[(a^2-b^2)=(a-b)(a+b)\]
Example:
\[(49-x^2)=(7-x)(7+x)\]

\subsection{Difference and Sum of Cubes}
The Difference and Sum of Cubes methods are used to factor pairs of cubes in additive and subtractive binomial groups.
\[(a^3+b^3)=(a+b)(a^2-ab+b^2)\]
\[(a^3-b^3)=(a-b)(a^2+ab+b^2)\]


\section{Simple/AC Method}
This method is used to factor trinomial quantities of roughly the form: 
\[ax^2+bx+c\]
Split or dont split a into a multiple/s which:\\
when multiplied with two terms whose product is c, add up to b.\\\\
It is important to know that this method can be used with multivariable quadratics as well.\\

Example:
\[(2x+2)(x+2)\]
\[2x^2+4x+2x+4\]
\[2x^2+6x+4\]
\[(2x+2)(x+2)=2x^2+6x+4\]


\section{Grouping}
This method can be used to factor polynomials which cannot be factored with the other methods because not all of the factors conform to a singular method. You make use of the associative and commutative properties of a variety of addition and subtraction to group together factors into quanities which can be factored with the other methods, precedence rules apply here in what you should do first.\\\\ 
Example:
\[x^2+8x+y^2+y\]
\[(x^2+8x)(y^2+y)\]
\[x(x+8)y(y+1)\]
\[(xy)(x+8)(y+1)\]


\section{Order of Application}
The above methods of factoring MUST be applied in a specific order to produce a solution which is factored optimally; solutions which are not factored to the simplest solution will be not recieve credence. The ordering of method application is as follows.


\begin{enumerate}
  \item Greatest Common Factor
  \item Special Methods
  \item Simple/AC Method and Grouping
\end{enumerate}


\section{Higher Degree Polynomial Factoring}
High degree polynomial factoring makes use of synthetic division to find factors of a polynomial using trial and error to pick out of a set of possible factors. Once you have reached a quadratic, if you cannot factor it further with basic methods, it is best to excersise the quadratic formula in case the answer is complex.
\[x^3+2x^2+4x+10\]\\
To factor this kind of problem utilize the Rational Zero theorem. Are possible zeros are all the factors of $a_0$ and $a_n$:
\[a_0=10 \land a_3=1\]
the factors of $p=a_0=10$ are: 
\[\pm10,\pm5,\pm2,\pm1\]
the factor of $q=a_3=1$ is: 
\[\pm1\]
the possible zeros are: 
\[\pm10,\pm5,\pm2,\pm1\]

now we perform synthetic division with each of these possible zeros 
until we find the ones which divide with no remainder. Notably since the zero is the number we set the divisor to x-zero because zero-zero =zero which is what we're looking for. \\

Didn't work:
\[\frac{x^3+2x^2+4x+10}{x-10}\]
\[\begin{array}{r|rrrr}
  10 & 1 & 2  & 4   & 10 \\
     &   & 10 & 120 & 1240 \\ \hline
     & 1 & 12 & 124 & 1250
\end{array}\]\\\\

Unfortunately this problem is not solvable with this method, I tried all the possible zeros and they weren't valid, however this is a good example of how to solve a problem like this one. You have to use cardano's cubic method to solve this. 


\end{document}
